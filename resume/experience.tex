\cvsection{Experience}

\begin{cventries}
  \cventry
    {Linux Graphics Driver Engineer} % Job title
    {芯瞳半导体} % Organization
    {陕西, 西安} % Location
    {Sep. 2019 - PRESENT} % Date(s)
    {
      \begin{cvitems} % Description(s) of tasks/responsibilities
        \item {libGL: 开发 GPU 用户态驱动,实现对OGL 3.0 - 4.3的大部分API的支持 (2019-2020)}
        \item {libgbm: 移植 Mesa 实现的 GBM 后端, 以支持基于 Generic Buffer Manager 接口的应用和功能,如 Xorg 的 Glamor (2019-2020)}
        \item {GLX: 实现 GLX 1.4 API (2020-2021)}
        \item {DDX: 移植 xf86-video-modesettting, xf86-video-modesettting 作为 X11 服务器内置的 2D 显示用户态驱动,实现了大部分通用显示模式设置的功能,但由于其主要为了通用性设计,因此在移植后针对 GB 硬件做了性能优化,如 (2020-2021)}
        \item {Jenkins: 完成了UMD从编译,测试,打包(rpm包和deb包),发布全流程的CI搭建和自动化 (2021)}
        \item {Mesa/Gallium: 第2代显卡驱动基于 Mesa Gallium 框架,重新实现了 ARB\_geometry\_shader4,  ARB\_uniform\_buffer\_object, ARB\_transform\_feedback 等重点核心扩展,并进行相关性能调优 (2021-2023)}
        \item {JDX: 利用 LD\_PRELOAD 技术开发了公司级的调试工具,至今仍然是驱动问题定位的常用工具之一 (2022)}
      \end{cvitems}
    }

  \cventry
    {Software Engineer} % Job title
    {中兴通讯} % Organization
    {陕西, 西安} % Location
    {Jul. 2015 - Sep. 2019} % Date(s)
    {
      \begin{cvitems} % Description(s) of tasks/responsibilities
      \item {属于LTE 4G SON(自组织网络)团队,比较熟悉的 SON 模块的组件是 MLB(移动负载均衡), 期间主要是维护 SON 模块的相关组件, 也开发过几个 MLB 相关的新功能。特性实现主要使用C++11,单例模式,工厂模式,抽象工厂模式等 (2015-2018)}
      \item {负责 SON 子模块中多个功能的内外场故障的分析和修复 (2018-2019)}
      \end{cvitems}
    }
\end{cventries}
